\addchap*{Abstract}
\thispagestyle{empty}

Diese Bachelorarbeit beschäftigt sich mit dem Preisverhalten von Tankstellen. Es werden Preismuster in den Preisänderungen von Tankstellen gesucht, die auf regelbasiertes Verhalten zwischen einzelnen Konkurrenten hindeuten könnten. Dazu wird der Datensatz der Markttransparenzstelle für Kraftstoffe verwendet, welcher die Preisänderungen der letzten drei Jahre umfasst. Weil bereits begründete Vermutungen über die Form von derartigen Regeln bestehen, wird, anstatt der üblichen Algorithmen aus dem Bereich des maschinellen Lernens, ein auf dieses Problem zugeschnittener Algorithmus verwendet, um dieses Hintergrundwissen gewinnbringend nutzen zu können.\\ 
Die Ergebnisse zeigen, dass Regeln sicher erkannt werden können, sofern diese auch kontinuierlich angewandt werden. Zu selten umgesetzte Regeln lassen sich nicht von zufälligerweise korrelierenden Preisänderungen unterscheiden. Zudem gibt es einige sehr stark korrelierende Muster die auf Regeln hindeuten, wo jedoch nur indirekte Zusammenhänge - über die Einwirkungen dritter Tankstellen - bestehen. Zuletzt wird noch ein Ausblick auf Verbesserungsoptionen dieses ersten Ansatzes gegeben, mittels derer die Fehlerrate noch weiter gesenkt werden könnte. 