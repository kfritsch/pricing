%einleitung.tex
\chapter{Einleitung}
\label{sec:Einleitung}

\section{Geschichtlicher Hintergrund}
Kraftstoffpreise unterliegen nun schon seit längerer Zeit großen Preisschwankungen. Die hohe Homogenität und sinkende Nachfrage sind optimale Bedingungen für starken Konkurrenzdruck. Der Druck ist so hoch, dass die Anzahl an Tankstellen in Deutschland seit 1970 von gut 46.000 auf heute knapp 15.000 gefallen ist \citep*{PraTa}. Trotzdem ist die Anzahl an Preisänderungen in den letzten Jahren noch einmal merklich gestiegen. Während der Preis 2011 an Tankstellen jeweils durchschnittlich weniger als zwei mal pro Tag geändert wurde, sind sie inzwischen größtenteils bei um die sechs Änderungen pro Tag angelangt \citep*{Unity}. \\

%TODO: Check for Pricings per Day for each month since start\\

Mitgrund für diese jüngste Steigerung könnten dieses Mal jedoch die neuesten elektronischen Möglichkeiten sein. Während Tankstelleninhaber vor einigen Jahren noch ihrer Konkurrenz einen Besuch abstatten mussten um deren Preise auszukundschaften, reicht heute eine Suchanfrage im Internet. Und selbst diese ist seit der Einführung der Markttransparenzstelle für Kraftstoffe nicht mehr notwendig. Jede Tankstelle ist seit Ende 2013 verpflichtet, ihre Preisänderungen dem Bundeskartellamt für Kraftstoffe elektronisch zu übermitteln. Der auf diese Weise entstandene Datensatz  bildet die Grundlage dieser Arbeit.\\

\section{Pricing Strategien}
Tankstellen waren in den letzten Jahren immer wieder gezwungen, ihre Strategien dem immer schneller werdenden Markt anzupassen. Die neu eingeführte zentrale und unmittelbare Verfügbarkeit aller aktuellen Preise ermöglicht es Tankstellenbesitzern, ihre Preise innerhalb kürzester Zeit an die Konkurrenz anzupassen. Das Bestehen einer einheitlich definierten Schnittstelle ermöglicht es auch, Preisänderungen zunächst automatisch zu registrieren und anschließend die davon betroffenen eigenen Preise anhand von Regeln ebenfalls automatisch anpassen zu lassen. Neben den offensichtlichen Vorteilen, wie zum Beispiel der Einsparung von Arbeitskräften, birgt diese Automatisierung aber auch Probleme.\\
Solange Preise noch mit viel Aufwand manuell verglichen werden mussten, war es nur möglich einen sehr kleinen Kreis an anderen Tankstellen zu überwachen um mit deren Preisen zu konkurieren. Die Vereinfachung der Arbeit durch automatische Prozesse führt dazu, dass mehr Konkurrenten bei gleichzeitig geringerem Aufwand überwacht werden können. Zudem können die gewünschten Reaktionen viel zuverlässiger durchgeführt werden. Das führt zu einem generellen Anstieg an Preissenkungen und damit auch zu einer allgemeinen Senkung der eigenen Preise. Um dem entgegen zu wirken, müssen diese zu Anfang des Tages höher angesetzt werden. Inzwischen wurde sogar eine sogenannte Mittagserhöhung eingeführt, um dem Preisverfall entgegen zu wirken. Auch eine gewisse Abstimmung von Preisen zwischen den Konkurrenten wird immer wichtiger. Wollen zwei Tankstellen jeweils einen günstigeren Preis fahren als die jeweils andere, so würde der Preis kontinuierlich sinken, bis eine Tankstelle ihr Minimum erreicht hat. Es gibt unter Tankstellen deshalb so etwas wie eine Rangordnung, die den Markt in A-Preiser und B-Preiser unterteilt. A-Preiser, zumeist die Markentankstellen, erlauben den B-Preisern, zumeist freie Tankstellen oder kleinere Tankstellenketten, etwas geringere Preise zu fahren. Während der Preisverfall bei einem Verstoß gegen diese Rangordnung vor der Automatiserung noch relativ langsam von voranging, fällt der Preis bei automatischem Pricing innerhalb weniger Minuten auf das Minimum. Zudem müssen jegliche Art von Sonderfällen in den Regeln berücksichtigt werden, da schon kleine Fehler große Verluste erzeugen könnten. Um ungewünschten Preisen durch einen vollautomatischen Betrieb vorzubeugen, beschränkt sich der automatisierte Part des Pricings teilweise auf den Vorschlag von Preisänderungen, welche dann noch manuell durchgeführt werden müssen.\\

\section{Motivation}
Die Idee dieser Arbeit besteht darin, das Konkurrenzverhalten der Tankstellen zu analysieren. Es ist anzunehmen, dass eine Tankstelle, egal ob sie ihre Preise manuell oder automatisch verwaltet, versucht diese möglichst gewinnoptimierend zu gestalten. Preise werden demnach im Allgemeinen nach bestimmten Richtlinien und in Abhängigkeit von verschiedenen Parametern festgelegt. In den Fällen, wo Entscheidungen konsistent nach bestimmten Parametern getroffen werden, entsteht im Grunde regelbasiertes Verhalten. Das unterschiedliche Verhalten erlaubt im Gegenzug Rückschlüsse auf die zu Grunde liegenden Regeln. Je konsistenter eine Regel umgesetzt wird, desto einfacher ist es, diese anhand ihrer Auswirkungen zu rekonstruieren. Je automatisierter also eine Tankstelle ihr Pricing betreibt, desto einfacher sollte es sein, ihr Verhalten zu analysieren.\\

\subsection{Unmittelbarer Nutzen}
In einem Wettbewerb ist es immer von Vorteil, die Strategien der Konkurrenten zu kennen. Es ermöglicht eine akkuratere Vorhersage des zukünftigen Marktgeschehens und hilft dabei, die eigene Strategie gewinnbringender zu gestalten. Da das Pricing in großen Tankstellenketten zentral betrieben wird, liegt die Verantwortung für das Preisverhalten von mehreren Hunderten Tanstellen bei einem sehr kleinen Personenkreis. Angenommen diese Personen würde gerne das Verhalten ihrer Konkurrenten vorhersagen können und würden versuchen die Regeln eigenhändig zu ermitteln. Bei beispielsweise 100 Tankstellen mit jeweils fünf potentiellen Konkurrenten und sechs Preisänderungen pro Tag müssten diese Personen Regeln für 500 Konkurrenten in 3000 Preisänderungen pro Tag erkennen. Eine automatisierte Regelerkennung würde also viel Arbeit und Mühen ersparen und wahrscheinlich sogar verlässlichere Ergebnisse liefern. Die eingesparte Zeit kann dann in die Optimierung des eigenen Verhaltens investiert werden, wofür die verlässlicheren Ergebnisse eine viel bessere Grundlage darstellen.\\

\subsection{Potentielle Möglichkeiten}
Jede Preissenkung verringert die Gewinnspanne einer Tankstelle. Es liegt also prinzipiell nicht im Interesse einer Tankstelle, Preise zu senken. Angenommen, eine automatisierte Regelerkennung wäre gegeben. Es bestünden dann einige Möglíchkeiten, dieses Wissen gewinnbringend zu nutzen:
\begin{itemize}
\item Mit einer solchen Regelerkennung ließen sich reaktive Preisänderungen von aktiven unterscheiden. Einzelne Änderungen könnten somit iterativ über die jeweiligen auslösenden Konkurrenten zurückverfolgt werden bis hin zu einer aktiven Änderung, also dem eigentlichen Auslöser. Die Erkennung der Initiatoren von Preissenkungen bietet die Möglichkeit auf solche Änderungen anderweitig zu reagieren und Senkungswellen zu umgehen.
\item Ebenso denkbar wäre, dass es Auslöser in Form von aktiven Senkungen gar nicht so häufig gibt wie angenommen. Einige Tankstellen reagieren zwar bei ihren Preissenkungen auf ihre Konkurrenz, legen ihre Tagesstartpreise aber unabhängig fest und führen die übrigen Erhöhungen eigenständig durch. Eine Tankstelle, die ihre Tagesstartpreise weniger hoch ansetzt, wäre somit potentieller Auslöser von einer Senkungswelle, weil andere Tankstellen mit schon bestehenden Preisen darauf reagieren müssen. Es wäre also unter Umständen möglich, dass sich einige Tankstellen, auch in eigenem Interesse, intensiver mit den eigenen Erhöhungen beschäftigen müssten.
\item Es ist bereits ohne dieses Tool möglich, einige Preisänderungen grob über größere Strecken zurückzuverfolgen. Mit dem Tool bestünde die Möglichkeit, die genauen Verkettungen von Konkurrenten zu erkennen und so Preiswege exakt darzustellen. Man könnte dann versuchen, diese Ketten zu unterbrechen, sodass isoliertere Preissektoren entstünden und Senkenswellen kleinere Umfänge hätten.
\item Die Auswertung der Ergebnisse bezüglich der eigenen Tankstellen bietet auch die Möglichkeit, die eigenen Änderungen systematisch zu analysieren und und weniger wichtige Konkurrenten aus dem System zu nehmen.
\end{itemize}
Diese Änderungen müssen jedoch nicht zwangsläufig nur den Tankstellen zugutekommen. Eine Tankstelle ist gesetzlich verpflichtet, keine verlustbringenden Preise zu fahren, um Preisdumping zu vermeiden. Sie muss somit die ganzen Preissenkungen an einem Tag in die Preiserhöhung zu Tagesbeginn einkalkulieren. Das ist höchstwahrscheinlich auch ein Grund für die erst kürzlich eingeführte Mittagserhöhung. Eine Verminderung der Preissenkungen führt somit nicht zwangsweise zu höheren Preisen, da der Wettbewerb weiterhin besteht. Sie sollte vielmehr zu generell stabileren Preisen führen, sodass Kunden sich besser auf Preise einstellen können.