%anforderungsanalyse.tex
\chapter{Anforderungsanalyse}
\label{sec:Anforderungsanalyse}

Thema dieses Abschnittes ist es, das Ziel der Arbeit genau zu spezifizieren und zu modularisieren. Das heißt, das Ziel soll bereits in logische Untereinheiten unterteilt werden. Dafür müssen bereits absehbare Probleme erkannt und notwendige Zwischenergebnisse definiert werden.

\section{Zielsetzung}
Ziel ist es, in den Pricingdaten Verhaltensmuster zu erkennen. Es soll in dieser Arbeit ausschließlich um das Erkennen von reaktivem Verhalten gehen. Das heißt, es geht darum zu erkennen, nach welchen anderen Tankstellen sich eine Tankstelle in welchem Maße richtet. Dabei dient das Konzept des vorgestellten Tools als eine generelle Verhaltensrichtlinie. Es ist anzunehmen, dass Tankstellen, die nicht speziell dieses Tool verwenden, entweder ein sehr ähnlich gestaltetes Tool betreiben oder aber manuell nach ähnlichen Prinzipien operieren.\\
Das liegt an den Grundlagen des Wettbewerbes. Für eine Tankstelle geht es darum, einen möglichst hohen Umsatz an Kraftstofflitern zu einem möglichst hohen Preis zu erzielen. Die für den Kunden interessanten Faktoren sind hierbei in erster Linie der Preis und die Entfernung zur Tankstelle. Die Gewichtung dieser Faktoren kann sich zwischen verschiedenen Kunden und zu verschiedenen Zeiten unterscheiden. Das generelle Kundenaufkommen ist abhängig vom Verkehrsfluss, unterscheidet sich also täglich wie stündlich und teilweise auch saisonbedingt. Es sollte also im Interesse einer Tankstelle liegen, zu den für potentielle Kunden interessanten anderen Tankstellen einen konkurrenzfähigen Preisabstand zu wahren, der unter Umständen zeitlich variieren kann. Demnach sind die Parameter, nach denen in dieser Arbeit gesucht werden soll, die jeweiligen Konkurrenten einer Tankstelle, die zu diesen Konkurrenten gewahrten Preisabstände sowie der jeweilige Gültigkeitszeitraum eines Abstandes.\\
Ein weiterer Grund, sich an den Regeln der Tools zu orientieren besteht darin, dass an tatsächlich in den Tools verwendeten Regeln die Performance des verwendeten Ansatzes überprüft werden kann. Das bedeutet nicht, dass das Ziel dieses ersten Ansatzes bereits sein soll, die verfügbaren Regelsätze zu hundert Prozent nachzubilden. Es geht immer noch primär darum, die aus dem tatsächlichen Verhalten ersichtlichen Reaktionsmuster zu erkennen. Dass dies ein Unterschied ist, wird im weiteren Verlauf deutlich werden. Die Regeln sollen zunächst nur dabei helfen, die errechneten Ergebnisse auch überprüfen zu können.

\section{Generelle Voranalyse}
Im Bereich der Mustererkennung gibt es einige kritische Faktoren, welche besonders ausschlaggebend sind für das prinzipielle Vorgehen. Zunächst muss die Menge der Daten evaluiert werden. Während bei großen Datenmengen die Geschwindigkeit und der verbrauchte Speicher zu kritischen Faktoren werden, sinkt bei geringen Datenmengen schnell die Aussagekraft. Zudem haben Anzahl, Abgrenzung und Konsistenz der Muster einen großen Einfluss auf die benötigte Datenmenge. Es muss auch geklärt werden, welche Daten in welcher Form benötigt werden und wie mit fehlenden und fehlerhaften Werten umgegangen werden soll. Zuletzt müssen anhand der Auswertung der genannten Faktoren, der Art des Problems und der Verfügbarkeit von Trainingsdaten mögliche Lösungsansätze ermittelt werden.

\subsection{Datenmenge}
Der Datensatz besteht aus knapp 15000 Tankstellen und insgesammt gut 80 Millionen Preismeldungen, erfordert also aufgrund seiner Größe eine Vorstrukturierung. Eine erste Selektion der Daten ist allein durch die Struktur der gesuchten Muster bereits gegeben. Es werden preisliche Korrelationen zwischen jeweils zwei Tankstellen vermutet, was die Menge durchschnittlich auf  circa 5000 Preisänderungen pro Tankstelle reduziert. Hinzu kommt, dass die zu einer konkurrierenden Tankstelle gewählten Regeln nicht über Jahre hinweg bestehen bleiben müssen. In Anlehnung an die Funktionen des Pricing Tools wäre es möglich, dass Regeln nach einiger Zeit wechseln, nur für bestimmte Tage definiert sind oder nur für bestimmte Uhrzeiten.\\
Einmal angenommen, eine Tankstelle würde für ein halbes Jahr lang eine Regel nicht ändern, welche für drei Stunden an einem bestimmten Wochentag gültig ist. Dann würden durchschnittlich lediglich knapp 24 Änderungen in den relvanten Zeitraum fallen. Angenommen alle diese Änderungen seien  Preissenkungen und die Tankstelle hätte vier gleichwertige Konkurrenten. Dann wären es nurnoch sechs Preisänderungen, die dieser Regel entsprechen würden. Abgesehen davon, dass die Tankstelle auch noch die Möglichkeit hat, die eigenen Regeln zu ignorieren, wären das viel zu wenig Daten, um eine statistisch signifikante Aussage treffen zu können. Es muss schließlich auch noch berücksichtigt werden, dass auch genau diese sechs Änderungen als Reaktionen auf diesen Konkurrenten erkannt werden müssen.\\
Die hohe Flexibilität bei der Wahl des Zeitraumes für einen Preisabstand und die Möglichkeit Regeln nach belieben wieder zu ändern, machen es notwendig, ein paar Einschränkungen zugunsten signifikanterer Ergebnisse vorzunehmen.

\subsection{Strukturierung des Datensatzes}
Da die Datenanalyse in Python durchgeführt werden soll müssen die Daten also zunächst von der Datenbank importiert werden. Aufgrund der Menge der Daten sollten nur diejenigen importiert werden, die für die Analyse einer Tanstelle benötigt werden. Die Daten sollten so in Python abgespeichert werden, dass besonders einfach und schnell auf die benötigten Datensätze zugegriffen werden kann. Dazu muss herausgearbeitet werden, nach welchen Kriterien besonders häufig auf Daten zugegriffen wird. Es muss auch untersucht werden, ob zusätzliche Daten benötigt werden oder Datenfelder unnötig sind. Auch die jeweiligen Datentypen  der einzelnen Felder müssen eventuell geändert werden.\\
Da die Daten zu einer Preisänderung von den Tankstellen selber angegeben werden, weisen sie ein paar strukturelle Unterschiede auf, welche berücksichtigt werden müssen. Zunächst ist es möglich, pro Kraftstoffsorte eine eigene Preisänderung vorzunehmen, oder aber die Änderungen in Gruppen durchzuführen. Dann gibt es Tankstellen, die vor jeder Änderung den Preis zunächst erst einmal auf Null setzen, bevor sie ihn wieder auf den eigentlich gewünschten Stand anheben. Auch senken einige Tankstellen den Preis bei Tagesende auf Null herab. In einigen Fällen werden Preise für die selbe Kraftstoffsorte in kürzester Zeit mehrmals geändert. Alle diese Vorgehensweisen haben potentiell verschiedene Auswirkungen auf eine automatische Analyse. Es muss einerseits überprüft werden, ob die strukturellen Unterschiede unwichtig sind und im Vorhinein angeglichen werden können oder aber wichtige Informationen enthalten und deshalb bewahrt werden sollten. Dabei muss der ganze Verlauf des Programmes berücksichtigt werden, um Probleme im späteren Verlauf zu vermeiden. Das selbe gilt auch im Bezug auf den Umgang mit fehlerhafte, fehlende oder wenig sinnvolle Daten.\\

\subsection{Algorithmen und andere Ansätze}
Zunächst geht es bei dem Problem darum zu entscheiden, ob eine bestimme Tankstelle auf eine Andere reagiert oder nicht. Das macht es unter anderem auch zu einem bivalenten Klassifizierungsproblem. Eine Konkurrenz lässt sich anhand der Existenz von regelmäßigem reaktivem Verhalten erkennen. Der einfachste Ansatz bestünde darin ein Set aus möglichen Regeln zu erstellen und diese alle anhand der Daten auf ihre Gültigkeit zu untersuchen. Das ist jedoch nicht möglich, weil die Anzahl an möglichen Regeln durch den variablen zeitlichen Parameter viel zu groß ist. Die wirklichen Regeln müssen also in den Daten erkannt werden. Es müssten also zunächst diejenigen Preisänderungen bestimmt werden, die zu einer Regel gehören. Da die Regeln selber noch nicht bekannt sind, können zunächst einmal nur die Preisänderung  bestimmt werden, die potentiell die Reaktion auf eine andere Preisänderung darstellen. Es müsste nach Paarungen von potentiell auslösenden Preisänderungen und den dazugehörigen Reaktionen in den kompletten Datensätzen zweier Tankstellen gesucht werden. Dazu muss jedoch zunächst bestimmt werden, wie so eine Paarung definiert ist.\\

Eine Möglichkeit, Reaktionen zu definieren und anschließend zu klassifizieren wäre es, Methoden aus dem Bereich des maschinellen Lernens zu verwenden. Diese Methoden benötigen zur Definition eines Modells, anhand dessen über einzelne Paarungen entschieden werden kann, klassifizierte Trainingsdaten, das heißt Paarungen bei denen bekannt ist, ob diese Reaktionen sind oder nicht. Als Trainingsdaten stehen jedoch nur wenige exemplarische Regeln eines einzigen Tankstellenunternehmens zur Verfügung. Man könnte mittels dieser Regeln versuchen einige Reaktionen aus dem Datensatz zu extrahieren. Das Ergebnis wäre jedoch kein repräsentativer Auszug aus dem Datensatz, da einerseits zu wenige Regelsätze zur Verfügung stehen und andererseits nur das Verhalten eines einzigen Anbieters enthalten ist. Das Modell würde also selbst wenn die Menge der Trainingsdaten ausreichend wäre nur die Charakteristiken einer Tankstelle lernen und wäre damit nicht auf die anderen Tankstellen übertragbar.\\

Alternativ käme in Frage die Reaktionen als kausale Ereignisse zu betrachten und die aus den Zeitstempeln erstellbaren Zeitreihen mittels statistischer Methoden auf Korrelationen zu untersuchen. Zur Analyse von Kausalität in Zeitreihendaten werden oftmals verschiedene Variationen des Granger Causality Tests verwendet. Bei diesem Test wird generell geprüft, ob Ereignisse aus zwei Zeitreihen in kausaler Relation stehen. Zeitreihe A wäre nach diesem Test kausal abhängig von Zeitreihe B, wenn die Vorhersage des jeweils nächsten Ereignisses in Zeitreihe A signifikant verbessert werden würde, wenn man als Parameter neben den vorherigen Werten von A auch die vorherigen Werte aus der Zeitreihe B hinzunimmt. Für einen solchen Test müssten zunächst Zeitreihen aus den Preisänderungen generiert werden. Da Reaktionen innerhalb von kürzester Zeit erfolgen können, müssten geringe Zeitintervalle gewählt werden. Das würde zu Zeitreihen mit fast ausschließlich gleichen Werten führen, weil der Preis oft über mehrere Stunden hinweg gleich bleibt. Ein Vorhersagemodell, das immer den gleichen Preis vorhersagt, wäre also nicht mehr signifikant zu übertreffen, weil es bereits eine Trefferquote von annähern 100\% besäße. Man könnte versuchen, dieses Problem zu umgehen und die Zeitintervalle, in denen keine Änderungen stattfindet, ausschneiden. Selbst wenn eine Konkurrenz besteht sind aber nicht alle Preisänderungen Reaktionen auf nur eine Tankstelle. Man müsste also die oben beschriebene Selektion von potentiellen Reaktionspaarungen trotzdem durchführen. Selbst dann bestünde noch das Problem, dass mehrere verschiedene Regeln für die Reaktionen verantwortlich sind. Der Algorithmus prüft jedoch nur, ob Zeitreihen generell korrelieren. Man müsste die Reaktionen also nochmals nach den möglichen Abstandswerten für Regeln unterteilen. \\

Bei dieser Vielzahl an Faktoren, nach denen die Daten vorselektiert werden müssen, machen die meisten allgemein verwendeten Algorithmen wenig Sinn. Da es ohnehin notwendig ist die Daten auf die Problemstellung zugeschnittene vorzusortierung, ist es auch denkbar die Regeln mittels einfacher statistischer Mittel aus den Datengruppierungen zu ermitteln. Die Regeln sind inhaltlich über einen Preisabstand definiert. Es sind also nicht absolute Preise, sondern die Differenzen zu den Konkurrenten, die wirklich entscheidend sind. Es müsste also eine detailliert Analyse über diese Differenzen durchgeführt werden. Auch ist nicht jeder Differenzwert gleich interessant. Besonders wichtig sind nur die vergleichsweise hohen Preisabstände. Die Differenz wird aus dem Blickwinkel der reagierenden Tankstelle errechnet, wobei negative Werte bedeuten, dass diese Tankstelle einen niedrigeren Preis aufweist als der jeweilig betrachtete Konkurrent. Hohe Differenzen bedeuten für die reagierenden Tankstellen einen verhältnismäßig ungünstigen Preisabstand, wenn mit ungünstig hier die Attraktivität für Kunden gemeint ist. Der in der Regel festgelegte Preisabstand bildet also einen maximalen Schwellenwert, der nicht überschritten werden darf. Die maximalen Differenzen bedeuten also, sofern sie keine Ausreißer darstellen, den Schwellenwert der Regel. Solche logischen Überlegungen sind möglich, weil die Struktur der gesuchten Regeln bereits bekannt ist, können aber bestenfalls nur schwer in allgemeine Algorithmen eingebunden werden. Aufgrund dieser Überlegungen wird in diesem Ansatz eine problemspezifische Lösung erstellt.

\section{Modularisierung}
Aufgrund der großen Datenmenge sollten die Tankstellen einzeln analysiert werden, anstatt alle in einem Durchlauf abzuarbeiten. Auch wenn es eventuell möglich wäre, auch die Konkurrenten einer Tanstelle einzeln zu analysieren, ist es unter Umständen hilfreich, alle potentiellen Konkurrenten gleichzeitig zu untersuchen. Da nur jeweils eine Preisänderung pro eigenem Pricing als tatsächlicher Auslöser in Frage kommt, könnten so verschiedene potentielle Auslöser verglichen werden.
% Aus diesem Grund wird die Analyse einer Tankstelle auch nicht direkt zweiseitig durchgeführt. Damit wäre gemeint, für eine Tankstelle sowohl die Reaktionen auslösenden Tankstellen, als auch die auf diese Tankstelle reagierenden Tankstellen zu ermitteln. In Verbindung mit dem Ansatz alle potentiellen Konkurrenten gleichzeitig zu untersuchen würde eine zweiseitige Betrachtung zu größeren Verkettungen führen, da bei der Analyse der reagierenden Tankstellen auch wieder deren komplettes Set an möglichen Auslösern verglichen werden müsste. Deshalb werden nur die Reaktionen auslösenden Konkurrenten eines Standortes bestimmt.
Zudem kann ein Zeitraum angegeben werden, für den die Analyse einer Tanstelle durchgeführt werden soll.\\
Um nicht alle anderen 15000 Tankstellen anhand ihres tatsächlichen Verhaltens überprüfen zu müssen, sollte eine erste Vorauswahl getroffen werden. Alle wirklichen Konkurrenten auszuwählen hat dabei höhere Priorität als die Auswahl gering zu halten. Anschließend müssen für jede Preissenkung der untersuchten Tankstelle die möglichen Auslöser in den potentiellen Konkurrenten gesucht werden. Zudem werden bei jedem potentiellen Konkurrenten die Preisänderungen ermittelt, auf die nicht reagiert und mit den zu diesem Zeitpunkt gelten Preisen der untersuchten Tankstelle gepaart. Für die Summe aller Preispaarungen für den jeweiligen Konkurrent wird dann die Verteilung der Preisdifferenzen erstellt. Von den Maximalwerten abwärts wird dann auf mögliche Regeln überprüft.