%fazit.tex
\chapter{Fazit}
\label{sec:Fazit}

\section{Zusammenfassung}
Die Auswertung hat gezeigt, dass es prinzipiell möglich ist, vorhandene Regeln in dem Preisverhalten von Tankstellen zu erkennen. Sie hat jedoch auch gezeigt, dass der Datensatz dies nicht besonders einfach macht. Generell anwendbare Algorithmen zur Erkennung von Kausalität sind auf die Rohdaten nicht anwendbar und würden eine extensive Vorverarbeitung benötigen. Außerdem sind in Fällen, wo diese Algorithmen verwendet werden, die inhaltlichen Zusammenhänge, also die unabhängigen Faktoren, nicht bekannt, weshalb man sich auf die reine zeitliche Korrelation verlassen muss. Im Falle dieses Problems sind nicht nur die inhaltlichen Begebenheiten sehr klar definiert mit in erster Linie auch eher abhängige Faktoren, sondern es ist durch die Beschaffenheit der Pricing-Tools sogar schon klar, wie die Kausalität in Form von Regeln im Grunde aussehen kann. Da diese semantischen Informationen zur Verfügung stehen, liegt es nahe, diese in einem auf dieses Problem speziell zugeschnittenen Algorithmus zu verwenden. Eine gute Lösung für ein derart komplexes Problem zu entwerfen bedarf mehrerer Ansätze, die jeweils erneut getestet werden müssen und nimmt somit einige Zeit in Anspruch. Ein passendes Modell zu finden verläuft demnach eher nach einem evolutionären Prozessmodell.\\
Die vorliegende Arbeit übernimmt daher eher die Funktion eines ersten Prototypen. Dies ist auch ein Grund, warum noch nicht versucht wurde, die Parameter in den Ergebnissen zu optimieren. Der erster Prototyp dient hier in erster Linie der Einschätzung, ob eine automatisierte Erkennung in diesem Fall möglich ist, wie akkurat die Trefferquote sein könnte und wie aufwändig es wäre, diese zu implementieren. Die Auswertung der Ergebnisse hat hier gezeigt, dass es noch einige Parameter gibt, die eine verbesserte Klassifikation ermöglichen sollten und die verwendeten Parameter trotz erster Verbesserungen noch weiter optimiert werden müssen. Sie hat im Gegenzug aber auch gezeigt, dass es im Bereich der $\beta$-Fehler eine recht geringe Fehlerquote gibt, die statistisch wahrscheinlich nicht erkannt werden können. Im Bereich der $\alpha$-Fehler ist eine deutlich höhere Quote an Tankstellen zu vermuten, die eine zu starke Korrelation aufweisen, als dass eine kausale Einwirkung ausgeschlossen werden könnte. In gewisser Weise besteht dort auch eine kausale Abhängigkeit, die aber dann nicht direkt, sondern nur über eine dritte Tankstelle wirksam wird. Wie gravierend Fehlklassifikationen in diesem Fall sind, hängt dann stark davon ab, zu welchem Zweck die Ergebnisse eines solchen Programmes genutzt werden würden. Zudem wird im abschließenden Kapitel noch eine Möglichkeit vorgestellt, wie auch in diesen Fällen die Fehlerquote noch gesenkt werden könnte. Es soll dabei helfen, die tatsächlichen Auslöser eines Pricings von Korrelationen zu unterscheiden. Die in der Zielsetzungen formulierte Anforderung, im Pricingverhalten eindeutig erkennbare Muster zu identifizieren, ist wie in den Ergebnissen zu sehen, größtenteils erfüllt und sollte mit der Hinzunahme von den zusätzlich ermittelten signifikanten Parametern noch verbessert werden können. Inwiefern auch weniger eindeutige Muster eindeutig klassifizierbar sind, müssen dann tiefergehende Ansätze zeigen. Dazu werden im abschließenden Kapitel noch einige Möglichkeit vorgestellt, wie solche Ansätze aussehen könnten.\\
Einen stark limitierenden Faktor stellen die sehr wenigen und nicht repräsentativen Testdaten dar. Ohne bessere Testdaten ist es schwierig, die Performanz abzuschätzen und die Parameter zu optimieren. Auch hierzu wird noch eine Möglichkeiten vorgestellt, wie auch ohne Angaben von anderen Tankstellen ein gutes Trainings- und Testszenario geschaffen werden könnte.\\

\section{Ausblick}
In diesem letzten Abschnitt sollen noch ein paar weiterführende Überlegungen vorgestellt werden, die sich im Zuge dieser Arbeit ergeben haben und als Verbesserungen vorgesehen sind, aber in der für die Auswertung verwendeten Version noch nicht implementiert werden konnten.

% \titlespacing{\subsubsection}{0pt}{12pt plus 4pt minus 2pt}{-6pt plus 2pt minus 2pt}
% \subsubsection{Ausnahmen einbinden}

\subsection{Auslöser und Korrelation unterscheiden}
Ein großes Problem der aktuellen Version ist, dass eine einzelne Preisänderung mehrere potentielle Auslöser hat, wovon jeder einzelne jeweils in den Konfidenzwert für die zugehörige Tankstelle eingeht. Es kann somit vorkommen, dass in der Summe aller gefundenen Konkurrenten insgesamt mehr Auslöser zusammenkommen als tatsächlich Reaktionen bestehen. Jede Preisänderung kann natürlich in Wirklichkeit nur einen Auslöser haben. Durch diesen Umstand werden nicht nur die tatsächlichen Auslöser, sondern auch bloße Korrelationen in der Klassifikation verwendet und verursachen die vielen $\alpha$-Fehler. Um das zu verhindern, sollen die potentiellen Auslöser einer Reaktion verglichen werden, und der wahrscheinlichste soll ausgewählt werden. Es besteht natürlich die Möglichkeit, die richtigen Auslöser so auszuschließen und das Ergebnis somit zu verschlechtern. Allein die inhaltlichen Faktoren einer Preisänderung, also die geänderten Kraftstoffsorten, Höhe und der Zeitabstand reichen nicht aus, um den Auslöser eindeutig zu identfizieren. Dieser Schritt soll in einer zweiten Analyse nach der Erstellung der Regeln durchgeführt werden. Deshalb ist auch die Minimierung der $\beta$-Fehler so wichtig, da Tankstellen, die in der ersten Runde ausgeschlossen wurden, nicht wieder mitaufgenommen werden. In der zweiten Runde werden dann nur die den Regeln entsprechenden potentiellen Auslöser nochmals untersucht. Auch hier gilt es wieder, vorher möglichst keine Regeln ausgeschlossen zu haben.\\
Für die Unterscheidung von wahren Auslösern und Korrelationen soll nun ausgenutzt werden, dass ein sehr großer Anteil der Preisänderungen einer Tankstelle reaktiv sind. Selbst Initiatoren einer Preissenkung ändern ihre Preise im Grunde, um sich selbst günstiger als die Konkurrenz darzustellen, sind also auch eine Reaktion auf die Konkurrenz. Diese Initiationen sind unter Umständen nicht an einem einzelnen Konkurrenten ausgerichtet, sollten aber nur einen sehr geringer Anteil an allen Preisänderungen darstellen. Es mag einige reaktive, nicht regelbasierte Preisanpassungen geben, die schwieriger zu erkennen sein werden, aber der Hauptanteil, zumindest bei großen Markentankstellen, sollten eindeutige regelbasierte Reaktionen sein.\\
Unter den Konkurrenten wird also eine Untergruppe gesucht, die die Anzahl der Reaktionen annähernd vollkommen erklären kann. Bei Preisänderungen mit nur einem potentiellen Auslöser ist die dazugehörige Tankstelle mit ziemlicher Sicherheit in dieser Gruppe. Wenn alle potentiellen Auslöser einer Tankstelle sich mit Preisänderungen von Tankstellen überschneiden, die sich bereits in der Gruppe befinden, könnte die Tankstelle ausgeschlossen werden. Es würde nur zu Problemen kommen, wenn sich Tankstellen mit ihren Preisänderungen komplett überschneiden, weil sie zum Beispiel aufeinander reagieren. Dann wäre es wiederum möglich, die Tankstelle mit den früheren Änderungen als Konkurrent zu erachten, weil deren Preisänderungen mindestens indirekt die Auslöser der Reaktionen sind. Zudem könnte man in solchen Fällen auch überlegen, die Geolocations noch einmal auszuwerten. Liegen die beiden grob in einer Richtung, könnte man auch die näher gelegene präferiert auswählen. Es sollte mit dieser Erweiterung aber zumindest möglich sein ,weniger konsistent korrelierende Tankstellen im Umfeld als Konkurrenten auszuschließen. 

\subsection{Gegenrichtung überprüfen}
Eine Möglichkeit, Regeln noch einmal auf ihre Gültigkeit zu überprüfen wäre es, das Konkurrenzverhalten in die Gegenrichtung zu bestimmen. Konkurrenz besteht in erster Linie durch geteiltes Kundenaufkommen, sollte demnach also eigentlich auf beiden Seiten ähnlich aufgefasst werden. Es wäre allerdings nicht möglich, dass die Summe der Abstandswerte der jeweiligen Regeln negativ ist. Würde Tankstelle A, um einen Cent günstiger sein wollen als Tankstelle B, B aber wiederum höchstens den gleichen Preis dulden, dann würde der Preis innerhalb kürzester Zeit auf ein Minimum herabfallen. Bestimmte Regelkonstellationen schließen sich also gegenseitig aus.

\subsection{Generelle Muster bei Markentankstellen überprüfen}
Die großen Markentankstellen machen einen großen Teil des Marktes aus und besitzen teilweise mehr als 1000 Tankstellen in Deutschland. Bei jeder dieser Tankstellen für jeden Konkurrenten einzeln die Regeln abzuwägen wäre erhebliche Arbeit. Man könnte untersuchen, ob zu bestimmten Gruppen von anderen Tankstellen, zum Beispiel zu anderen Marken, generell gleiche oder zumindest sehr ähnliche Regeln gewählt werden, oder ob zum Beispiel nur bestimmte Gruppen überhaupt als Konkurrenten in Frage kommen.

\subsection{Marktsimulation}
Um die Ergebnisse besser testen zu können, könnte man versuchen, künstlich einen Markt zu simulieren. Es bieten sich hier zwei verschiedene Vorgehensweisen an. Man könnte einen künstlichen frei erfundenen Markt implementieren. Dieser sollte dann genau die Breite an verschiedenem Verhalten aufweisen, das auch im wirklichen Markt vermutet wird. Dazu müssten dann einige Annahmen über die tatsächlichen Verteilungen vom Verhalten im Markt getroffen werden. An diesen künstlichen Daten könnte dann getestet werden, welches Verhalten von dem Programm erkannt werden kann und wo Probleme auftreten.\\
Alternativ dazu könnte man auch versuchen, den realen Markt so gut es geht nachzubilden und darauf basierend Vorhersagen über neuerliche Preisänderungen zu treffen. Die wirklichen Preisänderungen könnten dann dazu verwendet werden, das Modell nach und nach mit zu verbessern. Dieser Ansatz kann natürlich nur funktionieren, wenn der ganze Markt generell stark regelbasiert ist und nur wenige Änderungen impulsiv oder zufällig durchgeführt werden. 